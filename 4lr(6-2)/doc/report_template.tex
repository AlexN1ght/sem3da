\documentclass[12pt]{article}

\usepackage{fullpage}
\usepackage{multicol,multirow}
\usepackage{tabularx}
\usepackage{ulem}
\usepackage[utf8]{inputenc}
\usepackage[russian]{babel}


\begin{document}

\section*{Лабораторная работа №\,4 по курсу дискрeтного анализа: поиск подстроки в тексте}

Выполнил студент группы М8О-207Б МАИ \textit{Цапков Александр}.

\subsection*{Условие}

\begin{enumerate}
\item Необходимо реализовать поиск одного образца-маски: в образце может встречаться «джокер» (представляется символом ? — знак вопроса), равный любому другому символу. При реализации следует разбить образец на несколько, не содержащих «джокеров», найти все вхождения при помощи алгоритма Ахо-Корасик и проверить их относительное месторасположение.
\item Вариант алфавита: Числа в диапазоне от $0$ до  $2^32 - 1$.
\end{enumerate}

\subsection*{Метод решения}

Для выполнения поиска задонного образца со встречающимися джокерами необходимо реализовать поиск множественных образцов с помощью алгоритма Ахо-Карасика. По сути это "многомерный" вариант алгоритма Боера-Мура, где sp функция определена не для одной строки-паттерна, а состовляется дерево типа trie, где для каждого суффикса у любого задонного патерна ищется самый большой префикс из любого другого паттерна которого мы ищем. вместо обычного сдвижения теперь используется переход по так называемым "связям неудач". Кроме связей неудач в этом trie присутствуют также "связи выхода". Они были добавлены в алгоритм создателям, чтобы решить проблему того, что иногда, если один паттерн являлся подстрокой другого паттерна, то алгоритм не давал его вхождения. связи же неудач указывают как раз эти паттерны подстроки, для того чтобы не потерять их в решении. 

Для поиска же именно шаблона с джокерами вводится массив, в который мы добовляем вхождения наших подстрок. если же в этом массиве на n-ной позиции оказалось число равное количеству паттернов на которое мы разбили изначальный рбразец, значит с позиции и начинается вхождение нашего образца в текст.

\subsection*{Описание программы}

Моя программа состоит из 2-х файлов: AC.hpp с реализацией trie нужного для алгоритма Ахо-Карасика и основного файла программы main.cpp. Клас с деревом реализован через template, для возможного дальнейшего использования (это шаблон для типа алфавита). Но сам поиск был реализован в мэйне. Это было сделано для оптимизации поиска по времени и памяти. так как размер текста может быть любым я не запоминаю его полностью, а сразу анализирую при вхождении, так как алгоритм работает таким образом, что проверяет текст посдедовательно по одному символу и никогда не возвращается назад. таким образом при вхождении нового символа текста я начинаю его сразу обрабатывать и уже потом перехожу к следующему, забывая о предыдущем. Но все же некоторую информацию мне запоминать пришлост: это "координата" позиции и сама позиция в тексте. координаты для всех позиций я запоминаю из-за требований к выводу программы, где нужно указать строку и позицию в этой строке, количество символов в строке может разниться и не определено, таким образом невозможно просто вычислить позицию не прибегая к использованию какого-либо буффера, поэтому я просто при считывание текста запоминал "координату" для каждой позиции и при нахождении конечного вхождения паттерна выводил эти координа ты по позиции на ала паттерна.

\subsection*{Дневник отладки}

Что и когда делали, что не работало, как чинили.
Этот пункт обязательный, если если было сделано несколько
посылок. Кратко опишите суть проблемы и способ ее устранения.

\subsection*{Тест производительности}

Померить время работы кода лабораторной и теста производительности
на разных объемах входных данных. Сравнить результаты. Проверить,
что рост времени работы при увеличении объема входных данных
согласуется с заявленной сложностью.


\subsection*{Выводы}

Описать область применения реализованного алгоритма. Указать типовые
задачи, решаемые им. Оценить сложность программирования, кратко
описать возникшие проблемы при решении задачи. 

\end{document}

